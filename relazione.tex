\documentclass[12pt,a4paper,oneside]{book}
\usepackage[italian]{babel}
\usepackage{emptypage}
\usepackage[utf8]{inputenc}
\usepackage{amsmath}
\usepackage{geometry}
\usepackage{graphicx}
% inserisci altre cartelle dove sono presenti le immagini
\graphicspath{{immagini/}}

\usepackage{calc}
\usepackage{xcolor}
\usepackage{tabularx}
\usepackage{float}

\usepackage{hyperref}
\hypersetup{
    colorlinks=true,
    linkcolor=black,
    filecolor=magenta,      
    urlcolor=cyan,
    citecolor=black,
    % Imposta Metaparametri per il PDF generato
    % Non lasciare righe bianche: è un errore sintattico
    % Imposta l'autore del PDF
    pdfauthor={AUTORE},
    % Imposta l'argomento della Relazione
    pdfsubject={ARGOMENTO},
    % Imposta il titolo del PDF
    pdftitle={TITOLO},
    % Imposta le parole chiave per il documento
    pdfkeywords={PAROLE, CHIAVE}
}

\usepackage[italian]{cleveref}

\usepackage[backend=biber,
style=alphabetic,
sorting=ynt]{biblatex}
\addbibresource{bibliografia.bib}


% Decommenta per xltabular
%\usepackage{xltabular}
% Decommenta per usare note a pie pagina in tabelle
%\usepackage{tablefootnote}

% Decommenta per usare pseudocodice
%\usepackage{algorithm}
%\usepackage{algorithmic}
%\floatname{algorithm}{Algoritmo}
%\renewcommand{\listalgorithmname}{Elenco degli Algoritmi}

% Decommanenta per usare appendici
% \usepackage{appendix}

% Decommenta per inserire programmi
%\usepackage{listings}

% Decommenta per usare tikz
%\usepackage{tikz}
%\usetikzlibrary{automata, positioning, arrows, shapes}


\begin{document}

    % Conteggio pagine con numeri romani
    \frontmatter

    %pagina del titolo
    

\begin{titlepage}

\newcommand{\HRule}{\rule{\linewidth}{0.5mm}} 

\newcommand{\UniversityName}{\textsc{\LARGE UNIVERSITÀ o ORGANIZZAZIONE}}

\center 
% Inserisci qua il logo di Unversità o Organizzazione
%\includegraphics[width=\widthof{\UniversityName}]{}\\[1cm]

\UniversityName\\[1cm]
\textsc{\Large Nome del Corso}\\[0.5cm] 

\HRule \\[0.4cm]
{\Huge \bfseries Titolo Progetto}\\[0.4cm]
{\Large \bfseries Sottotitolo Progetto}\\
\HRule \\[1.5cm]

\begin{minipage}[t]{0.4\textwidth}
\begin{flushleft} \large
\emph{Autore:}\\
Nome \textsc{Cognome}\\
\emph{Matricola:}\\
XXXXXX
\end{flushleft}
\end{minipage}
~
\begin{minipage}[t]{0.5\textwidth}
\begin{flushright} \large
\emph{Docente:}\\
Nome \textsc{Cognome}\\
\end{flushright}
\end{minipage}\\

\vfill

\rule{0.8\textwidth}{0.4pt}
\begin{center}
    % specifica l'anno accademico 
    {\large Anno Accademico YYYY/YYYY}
\end{center}

\end{titlepage}



  
    %Indice principale 
    \tableofcontents
    
    % Deommenta per indice programmi
    %\lstlistoflistings
    
    % Decommenta per indice pseudocodice
    %\listofalgorithms

    % Indice delle tabelle
    \listoftables
    
    % Indice delle figure
    \listoffigures

    % Conteggio pagine con numeri arabi
    \mainmatter

  

\end{document}